\documentclass[14pt,a4paper]{extarticle}

\usepackage{graphicx} % Required for inserting images


\usepackage{minted} % Модуль для вставки кода в документ
\setminted{fontsize=\footnotesize}

\usepackage[english, russian]{babel}
\usepackage{fontspec}
\usepackage[T2A]{fontenc} % Русский язык
\usepackage[utf8]{inputenc}


\setmainfont[Ligatures=TeX]{Times New Roman} %Шрифт
\setmonofont{Consolas}

\usepackage[bookmarks=true, colorlinks=true, unicode=true, urlcolor=black, linkcolor=black, anchorcolor=black, citecolor=black, menucolor=black, filecolor=black]{hyperref}


\graphicspath{{./img/}} % Путь до изображений

\usepackage{indentfirst}
\usepackage[left=2.5cm, right=1.5cm, top=2cm, bottom=2cm]{geometry} %Поля
\linespread{1.5} % Межстрочный интервал
\setlength{\parindent}{1.25cm} %Абзацный отступ



\title{Реферат введение в специальность}
\author{Соловьев Артем Сергеевич}
\date{Апрель 2024}

\begin{document}
	
	
	{ \fontsize{14 pt}{10pt} \selectfont
	\pagestyle{empty}

	\begin{center}
		МИНОБРНАУКИ РОССИИ
		  
		Федеральное государственное бюджетное образовательное учреждение высшего образования
		
		«САРАТОВСКИЙ НАЦИОНАЛЬНЫЙ ИССЛЕДОВАТЕЛЬСКИЙ ГОСУДАРСТВЕННЫЙ УНИВЕРСИТЕТ ИМЕНИ Н.Г. ЧЕРНЫШЕВСКОГО»
		
	\end{center}

\vspace{2 cm}

\begin{flushright}
	Кафедра информатики и программирования
\end{flushright}

\vspace{2 cm}

\begin{center}
	Реферат
	
	Введение в микроконтроллеры
\end{center}
\vspace{5 cm}
\begin{flushleft}
	студента 1 курса 151 группы
	
	направления 09.03.04 Программная инженерия
	
	факультета компьютерных наук и информационных технологий
	
	Соловьева Артема Сергеевича 
	
\end{flushleft}

\vspace{7 cm}
\begin{center}
	Саратов 2024
	
	
\end{center}
\newpage

\tableofcontents



\newpage

\pagestyle{plain}
	}
	
	\section{Введение}
	
	Сегодня мы живем в информационной эре, когда электронные устройства  окружают нас повсюду: банкоматы, бортовые компьютеры, принтеры, бытовые приборы, наши телефоны и персональные компьютеры. Всё это было бы невозможно без создания и развития однокристальных микроконтроллеров. 
	
	Микроконтроллер - это специальная микросхема, предназначенная для управления различными электронными устройствами и сочетающая в себе микропроцессор, периферийные устройства, ОЗУ и ПЗУ. Исходя из этого микроконтроллеры способны: 
	\begin{itemize}
		\item Выдавать напряжение
		\item Измерять напряжение
		\item Производить измерения
		\item Запоминать данные
	\end{itemize}
	
	Так как же вещь, которая, на первый взгляд, способна так мало, открыла новую эру в области применения компьютерной автоматизации? Каковы её истоки? Где чаще всего используют микроконтроллеры? Как, наконец, разрабатывать проекты на их основе и что для этого необходимо? Постараюсь ответить на эти вопросы в своей работе.
	
	\newpage
	\section{Основная часть}
	
	\subsection{История}
	
	В начале развития электроники для создания устройств требовались сложные схемы, состоящие из большого количества компонентов, такие как: резисторы, транзисторы, конденсаторы и так далее. Эти схемы были энергозатратными, огромными, требовали много проводников между собой, поэтому первые ЭВМ занимали целые комнаты и дома.
	
	Но с появлением интегральных микросхем в 60"=х годах прошлого века произошла революция, повлиявшая на развитие нашего мира. Огромные схемы удалось поместить на крошечный кусочек кремния. Сегодня на одном кристалле процессора может быть около миллиарда транзисторов, а размеры транзисторов измеряются в нанометрах.
	
	Две технологии, с помощью которой это стало возможно, называются фотолитографией и легированием. Легирование "--- внедрение в кремний атомов других элементов в кремний для создания одного транзистора, а фотолитография "--- процесс растворения фоторезистов, путем освещения его ультрафиолетом, что позволяет устанавливать места для легирования.
	
	Метод фотолитографии оказался очень удачным для производства электронных устройств. Сейчас даже обычные транзисторы изготавливают таким способом.
	
	Первым микропроцессором, состоящим из одного чипа, был Intel 4004, выпущенный в Америке в 1971 году. Однако это еще не был микроконтроллер в современном понимании. Чип не был независимым и требовал для работы внешнюю память и периферийный интерфейс, в виде других чипов. [\ref{comph}] % https://www.computerhistory.org/siliconengine/microprocessor-integrates-cpu-function-onto-a-single-chip/
	
	Первый же настоящий микроконтроллер был создан также в 1971, и им был TMS1000. Это был 4"=х битный микропроцессор с ROM и RAM памятью, шинами ввода/вывода, а также таймером. 
	
	К середине 70"=х Японская электронная промышленность уже начала производить микроконтроллеры для автомобилей. Включая 4"=х битные модели для автоматических дворников, электронных замков и 8"=ми битные для управления двигателем.  [\ref{2}] %https://web.archive.org/web/20190627082830/http://www.shmj.or.jp/english/trends/trd70s.html
	
	В то время большая часть микроконтроллеров имело или ROM, или PROM память, которую можно было запрограммировать лишь раз, после чего изменить прошивку было невозможно. Лишь в 1993 году была представлена EEPROM (электрически стираемая память), которая позволила не только многомиллионным компаниям, но и простым любителям применять микроконтроллеры в своих разработках.  [\ref{3}]  %https://spectrum.ieee.org/chip-hall-of-fame-microchip-technology-pic-16c84-microcontroller ::
	
	Однако порог входа все еще оставался высоким. Ведь что нужно для работы микроконтроллера? Необходим стабилизатор напряжения, чтобы контроллер получал всегда 5 вольт питания. Возможно понадобится кварцевый резонатор для стабильной частоты. Наконец, вам нужно загрузить программу с компьютера в чип, для чего нужен программатор. И это только минимальная обшивка, а если вы захотите хотя бы получать данные из вне? Все это усложняет работу простым любителям.
	
	Так было до появления платформы Arduino, которая запустила в массы использование микроконтроллеров в своих проектах. По сути Arduino "--- это просто микроконтроллер Atmega328, посаженный на удобную плату со всем необходимым для стабильной работы, немного измененный язык C++ и IDE для прошивки контроллера через usb.
	
	\begin{figure}[H]
		\centering
		\includegraphics[width = 0.5\linewidth]{nano}
		\caption[]{плата arduino nano}
		\label{nano}
	\end{figure}
	
	Секрет успеха Arduino в простоте. К платформе очень удобно подключаться, на ней есть всё необходимое и можно сразу приступить к разработке проектов.
	
	На сегодняшний день сфера микроконтроллеров является одной из самых перспективных в IT.
	
	По прогнозам компании Market Research Future (MRFR), мировой рынок микроконтроллеров вырастет с 10,39 млрд. долл. в 2022 году до 58,20 млрд. долл. к 2030 году с темпом роста 24,04 процентов в течение всего прогнозного периода (2023"=2030 гг.). MCU имеют разнообразные области применения, такие как промышленная автоматизация, корпоративные сети в центрах обработки данных, бытовая электроника, автомобилестроение и телекоммуникации.
	
	Все более широкое применение этих устройств в электронике открывает широкие возможности для роста производителей MCU. В последние годы спроса влияет увеличение спроса на микроконтроллеры со стор на микроконтроллеры для улучшения работы электронных устройств постоянно растет. Кроме того, на рост рынконы производителей медицинского оборудования, телекоммуникаций и автомобильной промышленности. [\ref{4}] 
	
	\subsection{Программирование микроконтроллеров } 
	
	На сегодняшний день существует большое количество семейств контроллеров. Вот список самых распространенных из них:
	
	\begin{itemize}
		\item MSP430
		\item ARM
		\item MCS 51
		\item PIC
		\item AVR
		\item Espressif
	\end{itemize}
	
	В основном для программирования контроллеров используются два языка: ассемблер и C, а для любительской разработки существуют языки aruino, mircopython. Для начала рассмотрим разработку на C.
	
	Для написания кода чаще всего пользуются такими IDE:
	
	\begin{enumerate}
		\item Atmel Studio
		\item CodeVisionAVR
		\item WinAVR
	\end{enumerate}
	
	После написания кода и компиляции, программа должна быть загружена или <<прошита>> в микроконтроллер. Для этого требуется специальное устройство "--- программатор. Его вход это COM или USB порт, по которому он подключается к компьютеру, а выход проводами к выводам микроконтроллера. Пример программатора "--- USBASP на рисунке \ref{usbasp}.
	
	\begin{figure}[H]
		\centering
		\includegraphics[width = 0.3\linewidth]{usbasp}
		\caption[]{Программатор USBASP}
		\label{usbasp}
	\end{figure}
	
	Для отладки программы даже не обязательно при малейшем изменении кода перепрошивать микроконтроллер. Есть специальная программа Proteus, которая эмулирует работу как самой платы, так и других радио деталей.
	
	А для сборки готового устройства необязательна пайка, ведь есть макетные платы, в которые можно подключать все элементы с помощью проводов как конструктор.
	
	
В заключении можно сказать, спрос на микроконтроллеры с годами будет только расти. Несмотря на это, в профессиональной среде требования для работы с ними довольно высокие. Программисту нужно разбираться в низкоуровневой работе компьютера, знать как работает память, иметь знания в схемотехнике.

Микроконтроллеры являются одним из самых прикладных направлений IT. Действительно, в нашей жизни они повсюду. Предел их возможностей ограничиваются лишь вашими навыками и воображением.
 
\newpage
\begin{center}
	\textbf{СПИСОК ИСПОЛЬЗОВАННЫХ ИСТОЧНИКОВ}
\end{center}
\begin{enumerate}
	\item\label{1}https://www.computerhistory.org/siliconengine/microprocessor-integrates-cpu-function-onto-a-single-chip/
\item\label{2}https://web.archive.org/web/20190627082830\newline/http://www.shmj.or.jp/english/trends/trd70s.html
	\item\label{3}https://spectrum.ieee.org/chip-hall-of-fame-microchip-technology-pic-16c84-microcontroller
	\item\label{4}https://www.icrowdru.com/2023/09/29/мировой-рынок-микроконтроллеров-mcu-к-2030/
	\item\label{5}https://habr.com/ru/articles/463669/ 
	\item\label{6}https://docs.oasis-open.org/mqtt/mqtt/v5.0/mqtt-v5.0.html
\end{enumerate}

\bibliographystyle{gost780uv}
\bibliographystyle{references}
\appendix
	
\end{document}
